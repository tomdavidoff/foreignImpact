\documentclass[12pt]{article}

\usepackage{amsmath}
\usepackage{graphicx}
\usepackage[margin=1in]{geometry}
\usepackage[style=authoryear,backend=biber]{biblatex}
\addbibresource{impact.bib}

\begin{document}

\title{Foreign buyers and local affordability: indicators from Vancouver}
\author{Thomas Davidoff\\Sauder School of Business, UBC}

\section{Introduction}

Taxes and restrictions on foreign purchases of residences have been implemented
in multiple jurisdictions worldwide with the stated purpose of making homes
more affordable for domestic residents. For example, in extending a ban on
foreign purchases of Canadian residential real estate, a government press
release stated: ``For years, foreign money has been coming into Canada to buy
up residential real estate, increasing housing affordability concerns in cities
across the country, and particularly in major urban centres. Foreign ownership
has also fueled worries about Canadians being priced out of housing markets in
cities and towns across the country.''\footnote{\textcite{gOC}.}

There are theoretical and empirical reasons to expect foreign buyer taxes to
succeed in reducing local housing prices, many of which are detailed in a
comprehensive study by \textcite{favilukisVanNieuwerburgh}. The disincentive to
purchase homes should reduce the number of buyers considering purchasing
properties and lower willingness to pay among remaining foreign buyers. This
reduction in prices hurts landlords and incumbent property owners through lost
property value and rents, but makes home ownership more affordable for locals
who do not yet own homes. There may be an associated loss of construction jobs,
and the analysis in \textcite{favilukisVanNieuwerburgh} does not consider the
effect of reciprocal taxes on domestic residents who may wish to purchase
property overseas.\footnote{It is not obvious that a target country would wish
	to reciprocate to disincentivize the host country's tax. For example,
	China, the source of most foreign buying in Canada prior to the recent
spate of taxes (per, e.g. \textcite{ctvNews}) has made efforts to discourage capital
flight.}

Existing studies of the impact of the entry and exit of foreign buyers on local home prices
present estimated effects that range from modest to quite large, as summarized
in \textcite{davidoffZheng}. \textcite{LiShenZhang},
\textcite{gorbackGlobalCapitalLocal2020}, \textcite{pavlovImmigrationFlows},
and \textcite{BadarinzaRamadorai} find that within metropolitan areas,
submarkets exposed to foreign buyers see larger price increases when the types
of foreign buyers prone to buy in those submarkets face increased incentives to
buy overseas. \textcite{DachisDurantonTurner}, \textcite{klevenBest},
\textcite{kopczukMunroe}, and \textcite{davidoffLeigh} find, as a general
matter, that increasing transaction taxes on housing purchases leads to fewer
transactions and lower prices. \textcite{HartleyForeign},
\textcite{andalfattoEstimatingEffectMetro2023}, \textcite{DuYinZhang}, and
\textcite{pavlovForeignBuyerTaxes} collectively find that foreign buyer taxes
in Australia, Canada, and New Zealand led to lower prices.

As \textcite{favilukisVanNieuwerburgh} observe, foreign buyers may not leave
homes empty, but rather rent them out to locals. In their calibration, this
changes the social welfare effect of foreign buyers from negative to slightly
positive. The effect of out of town buyers is moderate because local investors
are highly price sensitive in their demand for rental properties, and out-of-town
buyers represent a small share of overall capital investment. In this way,
foreign buyers represent a slightly positive form of capital accumulation. This
is a salient consideration, as British Columbia has imposed significant taxes on empty
homes and vacation properties in urban areas while also imposing significant
taxes on foreign buyers, and the Canadian federal government has both imposed
taxes on empty homes owned by foreigners and has temporarily banned the
purchase of single residences by foreigners.

At the time of writing, the market for condominiums in Vancouver and Toronto
are so weak that construction has stopped. While indicative of falling prices,
the lack of condo presales has led some observers to suggest that the taxation
of foreign buyers has overshot, leading to a negative supply response larger
than the positive demand response on prices from foreign buying, such that
locals have been adversely affected. It is thus worthwhile asking whether there
is empirical evidence supporting the idea that foreign buyers, particularly in
the presence of empty homes taxes, may have positive welfare effects and their
taxation adverse effects on housing affordability for locals. QUOTE GOODMAND AND RENNIE HERE

There are several channels through which foreign buyers could have positive
affordability effects. First, per \textcite{favilukisVanNieuwerburgh}, foreign
buyers may rent houses to locals, particularly in the presence of an empty
homes tax.\footnote{This observation is echoed in the BC context in a policy
brief, \textcite{Goodman}.} Second, foreign buyers may purchase condominiums in
the presale phase of condominium development and then assign their contracts
prior to completion. The presence of foreign presale purchasers would raise
presale prices, encouraging the supply of condominiums, but with no increase in
demand for occupied space. The net effect of that form of investment should be
positive for affordability for end users, with an adverse competition effect
for locals who wish to occupy homes purchased through the presale channel.

A third way in which foreign buyers could have a beneficial effect on
housing affordability for local apartment buyers if they have a relative
preference for quality, and if there is a high degree of vertical
differentiation within buildings. In that case, foreign buyers could make
projects more profitable than they would otherwise be, increasing supply, but
without crowding out local demand. This would be a variant of the positive
``pecuniary externality''

In the next sections, we review some available economic evidence on the extent to which 
the channels alluded to above are operative in the Vancouver housing market.

\section{A simple framework for evaluating foreign buyer effects on local housing afforadbility}

As discussed above, foreign buyers may or may not compete for the same units as
locals, and may rent the homes they own to locals. A simple model provides a
way to organize data to evaluate likely affordability effects of taxes and
quantity restrictions on foreign buyers.

Suppose that there is a very large number of potential foreign buyers with
infinitely elastic demand for housing units at a price of $p_{f}$, assumed
always higher than the equilibrium willingness to pay among locals that comes
from the demand curve $q_{l}\left(p_{l}\right)$. The fraction of buyers that
are foreign $\alpha$ in each building is then determined by a government
policy, here assumed to be exogenously
chosen.\footnote{\textcite{FavilukisVanNieuwerburgh} allow foreign demand to be
affected by the local price and by an time-varying exogenous factor.}

Developers charge different prices to foreign and local buyers. This could
arise through two channels. First, foreign buyers might demand higher quality
units, e.g. penthouses for which they have a much greater willingness to pay
than locals, but have relatively little interest in lower-floor, ordinary
units. Second, developers might market a fraction of presale units 
overseas, and the law of one price might not force equality of pricing if
foreign buyers are constrained in their ability to participate in ordinary
presale or resale markets.\footnote{An allegedly common practice, see, e.g.
\texttt{https://vancouver.citynews.ca/2017/06/06/developer-intend-give-overseas-buyers-first-shot-vancouver-project/}.}

The supply of housing units is given by a function $q_{s}\left(\bar{p}\right)$,
increasing in price. $\bar{p}$ is the average price at which homes are sold,
$\bar{p} \equiv \left[1-\alpha\right]p_{l} + \alpha p_{f}$.

Some fraction of housing units purchased by foreign buyers are rented out to
locals, and some presale buyers flip their homes prior to occupancy of
buildings. Call $z$ the fraction of units owned by foreign buyers that are
unavailable to locals. The magnitude of $z$ might be taken as a measure of how
important affordable home ownership is, beyond prioritizing renters. A large
$z$ value could thus indicate that local consumers renting from foreigners is
seen as undesirable by policymakers, whereas $z=0$ might indicate that
converting owner units to rental is not undesirable.\footnote{Indeed,
policymakers have implemented a range of policies designed to convert homes
from owner-occupied to rental, such as favorable financing from crown
corporation CMHC for rental housing and local zoning policies that allow more
density for purpose-built rental apartments than for condominium buildings.}

Equating supply and demand for local buyers yields the following equation:

\begin{equation}
	\label{eq:localSD}
	\left[1-\alpha z\right]q_{s}\left(\left[1-\alpha\right]p_{l} + \alpha p_{f}\right) = q_{l}(p_{l}).
\end{equation}

Developers will supply more units the higher the average price per unit paid by locals and foreign buyers weighted by their share of all units built. Implicitly differentiation of equation \eqref{eq:localSD} indicates that the effect on local prices of a small increase in the foreign buyer share depends is:

\begin{equation}
	\label{eq:implicit}
	\frac{dp_{l}}{d\alpha} = -\frac{-zq_{s} + \left[1-\alpha z\right]q_{s}'\left[p_{f}-p_{l}\right]}{\left[1-\alpha z\right]\left[1-\alpha\right]q_{s}'-q_{l}'}.
\end{equation}

The denominator of the right hand side of equation \eqref{eq:implicit} is positive as long as supply slopes up and local demand slopes down in price. The numerator of the right hand side of equation \eqref{eq:implicit} is negative, so that price falls with an increase in the permitted foreign share, if:

\begin{equation}
	\label{eq:sign}
	\text{sign}\left(\frac{dp_{l}}{d\alpha}\right) = \text{sign}\left(1-\underbrace{\frac{q_{s}'p_{l}}{q_{s}}}_{\text{supply elasticity}}\times\underbrace{\frac{p_{f}-p_{l}}{\bar{p}}}_{\text{foreign price premium}}\times\underbrace{\frac{1-\alpha z}{z}}_{\text{Domestic occupancy intensity}}\right).
\end{equation}

Summarizing the analysis, allowing a little more foreign buying will improve affordability when:

\begin{itemize}
	\item The supply of homes is highly responsive to prices.
	\item Foreign buyers pay much more than locals for apartments in the same buildings.
	\item Most units in new buildings are occupied by locals in a way that government appreciates, i.e. there are few foreigner owners ($\alpha$ is low), or foreign owners commonly rent to locals and that arrangement is not viewed as objectionable.
\end{itemize}

\section{Some evidence on the determinants of foreign buyer affordability effects}

\subsection{Supply elasticity}

Foreign buyers can improve affordability through their effective subsidy of new
supply. That effect will be stronger in markets in which supply is responsive
to prices. \textcite{paixao2021housing} provides a list of Canadian CMAs by
housing supply elasticity (estimated from one-year differences in prices and
quantities), and Figure \ref{fig:elasticityNonResident} plots that elasticity
against non-resident ownership.\footnote{As in \textcite{paixao202housing}, two
outliers with extremely high supply elasticites are omitted.} The non-resident
ownership data is for the year 2018 and comes from the Canadian Housing
Statistics Program (CHSP) table ``Residency participation of residential
properties, by property type and period of construction''.  The CHSP measure of
the fraction of property owners who do not reside in Canada is an imperfect
approximation of buyers who would be deemed foreign for tax purposes. The
absolutely and relatively modest supply elasticities in markets like Toronto
and Vancouver, where foreign buying is concentrated, suggest that foreign
ownership may have adverse afforability consequences if not paired with limits
on empty homes (moving $z$ towards zero in the context of equation
\eqref{sign}). It is worth remarking, though, that non-resident ownership was
modest in all large Canadian CMAs as of 2018, with Vancouver by far the highest
level at 6.3\%.

\begin{figure}
	\centering
	\includegraphics[width=0.8\textwidth]{"~/OneDrive - UBC/foreignImpact/text/elasticityOwnership.png"}
\caption{\label{fig:elasticityNonResident} Non-resident ownership and housing supply elasticity in Canadian CMAs. Sources: \textcite{paixao2021housing} and CHSP table ``Residency participation of residential properties, by property type and period of construction'' for 2018 estimates. (charted values of elasticities merged on CMA names by ChatGPT).}
\end{figure}

\subsection{Do foreign buyers pay more for units in the same building?}

Equation \eqref{eq:sign} suggests that foreign buyers can have a positive
supply effect when the units in which they stimulate supply are largely
occupied by locals (where $\alpha z$ is close to zero). If foreign buyers pay
higher prices than local primarily because they occupy different buildings than
locals, it is not easy to see a positive supply effect for
locals.\footnote{\textcite{FavilukisVanNieuwerburgh} account for the job
creation effects of the supply effect when foreign buyers purchase homes.}

Foreign buyers might pay more than locals for units in the same building either
by facing different prices for the same units, or by specializing in
particularly luxurious units within buildings. We cannot observe whether the
law of one price is violated, e.g.  through differential presale pricing
overseas.

There is some scope for answering the question of whether foreign buyers
purchase higher quality units within buildings than local buyers with available
data, at least for the resale market, if not for the critical presale market.
Figure \ref{fig:variance_decomposition} uses data taken from condominium
resales in Greater Vancouver between 2010 and 2023, provided by BC Assessment.
The plot measures the year of transactions on the horizontal axis. On the
vertical axis are measures of the variance of the natural logarithm of
transaction prices by year. The red line plots the variance of all transaction
prices. The green plots the variance of mean prices across
buildings.\footnote{If there were only one transaction for each building with a
transaction in a given year, the red and green lines would coincide.} The blue
line plots the mean variance of log transaction prices within buildings. 

The dashed vertical line in Figure \ref{fig:varianceDecomposition} coincides
with the imposition of the foreign buyer tax in July, 2016. Consistent with
foreign buyers demanding more expensive units (as non-resident buyers do in the
CHSP data),\footnote{See also}  the overall variance of prices drops sharply in
the years after the tax, consistent with loss of an important luxury segment of
the market.  However, the reduction in price variance appears to be almost
entirely driven by a reduction in price variance across buildings, rather than
within buildings. This suggests that foreign buyers did not play a large role
in within-building price variance.

\begin{figure}
	\caption{label{fig:varianceDecomposition} Variance decomposition of log resale transaction prices in Greater Vancouver, 2010-2023.}
\includegraphics[width=\textwidth]{"~/OneDrive - UBC/foreignImpact/text/variance_decomposition.png"}
\end{figure}

Figure \ref{fig:varianceDecompositionNew} provides a bit more support for the
notion that foreign buyers drive within-building price variation. There, data
are restricted to buildings less than five years old, as foreign buyers may
have been concentrated in new buildings (as non-resident buyers surely are) and
new building prices may be more salient to the presale pricing that presumably
drives developer supply choices.  While there is not a clear effect of the tax
\emph{per se}, it is clear that the share of all price variation explained by
within-building variance was growing prior to the implementation of the tax, at
the same time that foreign buyers was likely growing.

\begin{figure}
\caption{\label{fig:varianceDecompositionNew} Variance decomposition of log transaction prices in Greater Vancouver, 2010-2023, transactions in buildings less than five years old at time of sale (but resale transactions only) only.}
\includegraphics[width=\textwidth]{"~/OneDrive - UBC/foreignImpact/text/variance_decompositionNew.png"}
\end{figure}

Another indicator of within-building versus between price variation comes from
CHSP data comparing assessed values of properties owned by resident versus
non-resident owners. While we cannot observe building-level data, we can ask
what fraction of price differences in the value of units owned by non-residents
versus residents is explained by variation across municipalities, versus
within, and by non-residents' propensity to own newer homes. Overall, among condominiums in the Vancouver CMA, non-residents' have a mean value of \$790,000 versus \$630,000 for residents of Canada, a difference of over 25\%.

However, in a regression of log mean value by period of construction and municipality on dummies for municipality, so conditioning on broad location, the coefficient on non-resident owner versus resident owner is just .064, so the large majority of value is explained by location. Adding dummies for ranges of years in which buildings were completed reduces the estimated log difference between residents and non-residents to just .017. Thus, within buildings of the same age in the same general location, non-residents own units that are roughly identical in value to residents, leaving little scope for extra payment for fancier units (differences in price paid in preconstruction would not show up in these assessed values).

The fact that non-residents own newer units than residents might be seen as a form of differentiation in quality within buildings, in that if residents have less willingness to pay for new units than non-residents, non-residents may stimulate supply by paying more for new units that drive supply, leaving older units to resident buyers over time.\footnote{A caveat is that some of the difference in age of property owned may be attributable to non-resident purchases growing over time. If the order of controls is reversed, in a regression of log mean value on just period of construction dummies, the coefficient on non-resident ownership is -.065, so the difference in condo age explains as much of the value difference as municipality, and is unlikely to be mostly explained by differential dates of purchase.}

\section{Stock and flow of foreign purchases in Greater Vancouver}

For the years starting in 2018, the Canadian Housing Statistics Program has
used its access to ownership data on the universe of Canadian residences to
publish statistics on ownership by residency status in Canada. This is an
approximation of foreign ownership, because some owners who live overseas are
Canadian citizens or permanent residents, and hence exempt from foreign buyer
taxes. Also, some Canadian residents are not yet landed immigrants, and hence
are subject to the foreign buyer tax. Thus CHSP represents an approximation of
the stock of ``foreign'' ownership over time. Unfortunately, the stock measure
is not readily available prior to 2018. Between 2018 and 2022, the number of
non-resident owners in the Vancouver Census Metropolitan Area grew from 24,135
to 26,350, but fell as a share of all owners from 9.6\% to 9.0\%.

The apparent declining stock of foreign ownership as a share of all residences
(as proxied by nationality) is matched by a declining share of the flow into ownership. The B.C. provincial
government started collecting data on the nationality of residential property
buyers shortly before implementing the additional property transfer tax of 15\%
on foreign buyers in mid-2016. In the month (July) prior to implementation,
15\% of purchases in Metro Vancouver involved foreign participation. For the
years since, the fractions have been as shown in Table \ref{tab:fbt}. COVID and
the national foreign buyer ban have likely contributed to the decline over time
in foreign participation.

\begin{table}
	\caption{\label{tab:fbt} Fraction of residential transactions in Metro Vancouver with foreign participation, per BC Property Transfer Tax data. Source: \texttt{https://catalogue.data.gov.bc.ca}}
	\begin{tabular}{ll}
		\hline
		Year & Fraction of transactions with foreign participation \\
		\hline\hline
		2017 & 3.7\% \\
		2018 & 2.9\% \\
		2019 & 2.0\%\\
		2020 & 1.4\%\\
		2021 & 1.1\%\\
		2022 & 1.3\%\\
		2023 & 0.9\%\\
		2024 & 0.9\%\\
		\hline
	\end{tabular}
\end{table}

\section{Do foreign buyers leave homes empty?}

Owners of homes that are not primary residences are subject to extra taxes in
certain jurisdictions in Ontario and B.C., and foreign owners of Canadian homes
left vacant more than half of a year are subject to a federal ``Underused
Housing Tax.'' If these taxes are high enough, units will be rented or sold to
locals (such that $z$ is close to zero), or the tax revenue should be
sufficient to leave local residents better off than if the unit were owned by a
local. However, if empty home taxes are set at a sufficiently low level,
foreign owners may leave homes empty, and the tax revenue may not be sufficient
to compensate locals for the loss of housing stock.

In the presence of the B.C. empty homes tax regime, foreign, non-resident
owners who retain their properties most commonly rent them out to locals. Per
\textcite{specTax2019}, of 3,709 foreign owners deemed subject to the empty
homes tax in 2018, a large majority no longer held empty homes in 2019. 1,205
of these foreign owners subject to the Speculation and Vacancy tax transitioned
to renting their homes out in 2019, 1,413 sold their property, 951 remained
subject to the tax, and 66 transitioned to using the home as a primary
residence. Perhaps through ``burnout'' just over half of 1,308 foreign owners
deemed subject to the tax in 2020 remained subject to the tax in 2021. Overall,
for the 2021 tax year, roughly 5.5\% of declared foreign owners homes' were
deemed empty.

\subsection{Presale flips}

Another politically unpopular group of housing investors ... that is British Columbia has implemented an additional tax on profits from capital gains on homes held for less than two years. That tax applies also to flipping presale contracts

\section{Conclusion}

A simple model of foreign purchases reinforces natural intuitions that foreign
buyers can act to stimulate supply, and thus enhance affordability for locals
when the foreign buyers are a small part of the market, when they pay more than
much more than local residents would for homes in the same buildings, when
housing supply is elastic with respect to price, and when a large fraction of
foreign-owned homes are rented to locals. In this framework, preconstruction
purchases by foreign (or domestic) buyers followed by assignment to local
buyers prior to completion should have purely positive affordability effects.

Foreign buyers in Canada are concentrated in housing markets with relatively
low supply elasticities, and most of the variation in the value of apartments
owned by non-residents is explained by location and building age, not different
unit quality. These facts suggest that in the absence of empty homes taxes,
foreign buyers are likely to have little benefit to locals through a supply
channel, unless they face significant price discrimination in the
preconstruction market. However, in the presence of empty homes taxes, the
large majority of foreign owners appear to either use homes year round, rent
them to locals, or sell. Empty homes taxes thus seem likely sufficient to avoid
adverse affordability effects of foreign ownership.

\printbibliography

\end{document}
