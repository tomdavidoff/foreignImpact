\documentclass[12pt]{article}

\usepackage{amsmath}
\usepackage{graphicx}
\usepackage[margin=1in]{geometry}
\usepackage[authoryear,backend=biber]{biblatex}
\addbibresource{impact.bib}

\begin{document}

\section{Introduction}

Foreign buyer taxes have been implemented in multiple jurisdictions worldwide with the stated purpose of making homes more affordable for domestic residents. For example, in extending a ban on foreign purchases of Canadian residential real estate, a government press release stated: ``For years, foreign money has been coming into Canada to buy up residential real estate, increasing housing affordability concerns in cities across the country, and particularly in major urban centres. Foreign ownership has also fueled worries about Canadians being priced out of housing markets in cities and towns across the country.''\footnote{\textcite{gOC}.}

There are theoretical and empirical reasons to expect foreign buyer taxes to succeed in bringing down local housing prices. The disincentive to purchase homes should reduce the pool of buyers and lower willigness to pay among remaining foreign buyers. This reduction in prices is shown in a calibrated setting by \textcite{favilukisVanNieuwerburgh}, who find that overall social welfare for natives may increase with a foreign buyer tax, ignoring any countervailing measures. 

As \textcite{favilukisVanNieuwerburgh} observe, foreign buyers may not leave homes empty, but rather rent them out to locals. In their calibration, this changes the social effect of foreign buyers from negative to slightly positive. The effect of out of town buyers is moderate...

\end{document}
