\documentclass[12pt]{article}

\usepackage{amsmath}
\usepackage{graphicx}
\usepackage[margin=1in]{geometry}
\usepackage[authoryear,backend=biber]{biblatex}
\addbibresource{impact.bib}

\begin{document}
\title{Foreign Buyer Revisionism}

\section{Introduction}

Taxes and restrictions on foreign purchases of residences have been implemented in multiple jurisdictions worldwide with the stated purpose of making homes more affordable for domestic residents. For example, in extending a ban on foreign purchases of Canadian residential real estate, a government press release stated: ``For years, foreign money has been coming into Canada to buy up residential real estate, increasing housing affordability concerns in cities across the country, and particularly in major urban centres. Foreign ownership has also fueled worries about Canadians being priced out of housing markets in cities and towns across the country.''\footnote{\textcite{gOC}.}

There are theoretical and empirical reasons to expect foreign buyer taxes to
succeed in bringing down local housing prices, many of which are detailed in a
comprehensive study by \textcite{favilukisVanNieuwerburgh}. The disincentive to
purchase homes should reduce the pool of buyers and lower willigness to pay
among remaining foreign buyers. This reduction in prices hurts landlords and
incumbent property owners through lost property value and rents, but makes
homes more affordable for renters. There may be an associated loss of
construction jobs, and the analysis in \textcite{favilukisVanNieuwerburgh} does
not consider the effect of reciprocal taxes on domestic residents who may wish
to purchase property overseas.\footnote{It is not obvious that a target country
would wish to reciprocate to disincentivize the host country's tax. For
example, China, the source of most foreign buying in Canada prior to the recent
spate of taxes (per \textcite{ctvNews}) has made efforts to discourage capital
flight.}

Existing studies of the impact of the entry and exit of foreign buyers on local
home prices are mixed, with some finding large effects in the predicted
direction, as summarized in \textcite{davidoffZheng}. \textcite{LiShenZhang},
\textcite{gorbackGlobalCapitalLocal2020}, \textcite{pavlovImmigrationFlows},
and \textcite{BadarinzaRamadorai} find that within metropolitan areas,
submarkets exposed to foreign buyers see larger price increases when the types
of foreign buyers prone to buy in those submarkets face increased incentives to
buy overseas. \textcite{DachisDurantonTurner}, \textcite{klevenBest},
\textcite{kopczukMunroe}, and \textcite{davidoffLeigh} find, as a general
matter, that increasing transaction taxes on housing purchases leads to fewer
transactions and lower prices. \textcite{HartleyForeign},
\textcite{andalfattoEstimatingEffectMetro2023}, \textcite{DuYinZhang}, and
\textcite{pavlovForeignBuyerTaxes} collectively find that foreign buyer taxes
in Australia, Canada, and New Zealand led to lower prices.

As \textcite{favilukisVanNieuwerburgh} observe, foreign buyers may not leave
homes empty, but rather rent them out to locals. In their calibration, this
changes the social welfare effect of foreign buyers from negative to slightly
positive. The effect of out of town buyers is moderate because local investors
are so price sensitive in their demand for rental properties and out-of-town
buyers represent a small share of overall capital investment. In this way,
foreign buyers represent a slightly positive form of capital accumulation. This
is a relevant case, as British Columbia has imposed significant taxes on empty
homes and vacation properties in urban areas while also imposing significant
taxes on foreign buyers, and the Canadian federal government has both imposed
taxes on empty homes owned by foreigners and has temporarily banned the
purchase of single residences by foreigners.

At the time of writing, construction in Canadian cities, particularly
condominiums in the most expensive cities, Vancouver and Toronto, has slowed
dramatically. This has led some observers to suggest that the taxation of
foreign buyers has overshot, leading to a negative supply response larger than
the positive demand response on prices from foreign buying, such that locals
have been adversely affected. It is thus worthwhile asking whether there is
empirical evidence supporting the idea that foreign buyers, particularly in the
presence of empty homes taxes, may have positive welfare effects and their
taxation adverse effects on housing affordability for locals.

There are several channels through which foreign buyers could have positive
affordability effects. First, per \textcite{favilukisVanNieuwerburgh}, foreign
buyers may rent houses to locals. At least in the presence of the Provincial
and Vancouver empty homes taxes in B.C., over 90\% of foreign owners either
lived in or rented their property to locals. Of 3,709 foreign owners of empty
properties in 2018 Speculation Tax data, the top two means of exemption in the
following year were selling the property (1,413) and renting it out (1,205).
951 remained subject to the tax.\footnote{\textcite{specTax2019}.}

\begin{itemize}
	\item To what extent do foreign buyers leave homes empty, with and without empty homes taxes?
	\item Do foreign buyers far outbid local buyers for the most luxurious units in new buildings, and do those units drive profitability importantly in new construction?
	\item Did condo starts or presales underperform after foreign buyers were discouraged through empty homes and foreign buyer taxes and flip taxes?
\end{itemize}

\section{Do foreign buyers leave homes empty?}

\textcite{favilukisVanNieuwerburgh} find that foreign buyers do not leave homes empty, but rather rent them out to locals. This is consistent with the finding of \textcite{baker} that foreign buyers are more likely to be landlords than local buyers.

\section{Do foreign buyers specialize in the most luxurious units within buildings?}

Some 


\end{document}
