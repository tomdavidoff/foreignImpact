\documentclass[12pt]{article}

\usepackage{amsmath}
\usepackage{graphicx}
\usepackage[margin=1in]{geometry}
\usepackage[authoryear,backend=biber]{biblatex}
\addbibresource{impact.bib}

\begin{document}

\section{Introduction}

Taxes and restrictions on foreign purchases of residences have been implemented in multiple jurisdictions worldwide with the stated purpose of making homes more affordable for domestic residents. For example, in extending a ban on foreign purchases of Canadian residential real estate, a government press release stated: ``For years, foreign money has been coming into Canada to buy up residential real estate, increasing housing affordability concerns in cities across the country, and particularly in major urban centres. Foreign ownership has also fueled worries about Canadians being priced out of housing markets in cities and towns across the country.''\footnote{\textcite{gOC}.}

There are theoretical and empirical reasons to expect foreign buyer taxes to succeed in bringing down local housing prices, many of which are detailed in a comprehensive study by \textcite{favilukisVanNieuwerburgh}. The disincentive to purchase homes should reduce the pool of buyers and lower willigness to pay among remaining foreign buyers. This reduction in prices hurts landlords and incumbent property owners, but makes homes more affordable for renters. As renters is shown in a calibrated setting by , who find that overall social welfare for natives may increase with a foreign buyer tax, ignoring any countervailing measures. 

As \textcite{favilukisVanNieuwerburgh} observe, foreign buyers may not leave homes empty, but rather rent them out to locals. In their calibration, this changes the social welfare effect of foreign buyers from negative to slightly positive. The effect of out of town buyers is moderate because local investors are so price sensitive in their demand for rental properties and out-of-town buyers represent a small share of overall capital investment. In this way, foreign buyers represent a slightly positive form of capital accumulation.

The questions for foreign welfare benefits thus include:

\begin{itemize}
	\item To what extent do foreign buyers leave homes empty, with and without empty homes taxes?
	\item Do foreign buyers far outbid local buyers for the most luxurious units in new buildings, and do those units drive profitability importantly in new construction?
	\item Did condo starts or presales underperform after foreign buyers were discouraged through empty homes and foreign buyer taxes and flip taxes?
\end{itemize}

\section{Do foreign buyers specialize in the most luxurious units within buildings?}

Some 


\end{document}
